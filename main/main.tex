\documentclass[dvipdfmx,11pt,a4paper]{jsbook}
\usepackage{package}
\usepackage{a4wide}
\usepackage{tikz}
\usepackage{tikz-timing}
\usepackage{graphicx}
\usepackage{multirow}
\usepackage{listings}
\usepackage[dvipdfmx]{hyperref}
\usepackage{pxjahyper}

\lstset{%
language={C},
backgroundcolor={\color[gray]{.85}},%
basicstyle={\small},%
identifierstyle={\small},%
commentstyle={\small\itshape},%
keywordstyle={\small\bfseries},%
ndkeywordstyle={\small},%
stringstyle={\small\ttfamily},
frame={tb},
breaklines=true,
columns=[l]{fullflexible},%
numbers=left,%
xrightmargin=0zw,%
xleftmargin=3zw,%
numberstyle={\scriptsize},%
stepnumber=1,
numbersep=1zw,%
lineskip=-0.5ex%
}

\title{LATTICEEASY(日本語訳)}
\author{Haruto Yokoyama}
\date{\today}
\begin{document}
\maketitle

\tableofcontents

\makeatletter
\@addtoreset{equation}{section}
\def\theequation{\thesection.\arabic{equation}}
\makeatother

\chapter{概要}
LATTICEEASYはC発展する宇宙においてスカラー場との相互作用の格子シミュレーションをするC++のプログラムです。プログラムは簡単にパラメータを変えて動かすことができ、簡単に評価のために新しいモデルを挿入可能なようにデザインされています。LATTICEEASY2.0ではこれらのシミュレーションを1-3の次元で一つの変数を変えるだけで簡単に行うことができます。また、LATTICEEASYの並列処理のバージョンとしてCLUSTEREASYがあります。

プログラムはWebサイトの\url{http://www.science.smith.edu/departments/Physics/fstaff/gfelder/latticeeasy}にて使用可能です。これは自由に誰でも使い、修正することが可能です。詳細は第8章:基本的には我々のクレジットと連絡先を明記していただければ自由に変更を加えていただいて問題ありません。

もし何らかの質問やコメントがありましたら、gfelder@email.smith.eduまでご連絡ください。我々はあなたにとってプログラムがどんな風に動き、あなたが何らかの問題を解決できれば光栄です。バグや改善点などプログラムに関してご連絡いただければ幸いです。

この資料は4つのメインセクションに分かれています。4章はLATTICEEASYの使い方について記載されており、どのようにプログラムをコンパイルし、走らせるか、そしてどのように与えられたモデルに対して適切なパラメータをセットするか、最終的にどのように新しいモデルをプログラムを走らせるために作るかが記載されています。5章ではプログラムのアウトプットについて記載されています。そこにはアウトプットの関数と出力されるファイルに何が記載されているかを述べています。6章ではプログラムで用いる式について記載されています。このセクションの大半はプログラムを使う上で必要はありません。しかし、どのようにしてプログラムが動いているのか、何が起こっているのかを知る上では必要となってきます。このセクションで最も重要なパートは変数をプログラムでリスケールすることができることです。式を簡単にするためにプログラムは場や時空に対してリスケールされた値を用い、これらのリスケールについては6章にて説明がなされています。これらのリスケールは4章や5章を通して使われています。なのでそれらのセクションを読むときにより理解理解しやすくなるでしょう。7章はLATTICEEASYの並列処理について記載されています。

メインセクションに比べてマイナーなセクションがいくらかあります。2章は記法と慣習について、3章はLATTICEEASYファイルのリストとそれぞれが何をしているかについて説明しています。8章はプログラムの使用方法について。この"クレジット"セクションはlatticeeasy.cppの一番上に記載してください。



\bibliographystyle{unsrt}
\bibliography{reference}
\end{document}
